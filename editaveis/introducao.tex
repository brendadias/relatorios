\chapter{Introdução}

Os chefes de Estado presentes na Assembleia Geral das Nações Unidas de 2015 adotaram a Agenda para o Desenvolvimento Sustentável, sendo uma das metas a redução, pela metade, do número global de mortes e lesões relacionadas ao trânsito, até 2020 \cite{relatorio_oms}.

Todos os anos acidentes de trânsito matam no mundo cerca de 1,2 milhões de pessoas, sendo a primeira causa de mortes entre jovens entre 15 e 29 anos. O Brasil é o o quinto país com mais mortes no trânsito, registrando em torno de 47 mil mortos e 300 mil feridos graves por ano. Esses acidentes representam um custo de 40 bilhões de reais por ano \cite{relatorio_ipea}, sendo esse valor causado principalmente pela perda de produção, seguido de custos hospitalares.

\section{Contextulização}

No Brasil, o número de acidentes em curvas por tipo de traçado em 2017, ficou em segundo lugar em números de ocorrência, somente atrás de acidentes em linha reta. Consequentemente, o número de feridos graves e leves ultrpassaram mais de 3000 pessoas, ocupando a segunda colocação, somente atrás de acidentes em retorno regulamentado \cite{anuario_rodoviario}.

A principal causa para tantos acidentes e principal causa de óbitos nas estradas brasileiras é a falta de atenção à condução, sendo esta a responsável por, respectivamente, $38,5\%$ e $29,5\%$ do número de ocorrências \cite{anuario_rodoviario}. Isso pode ser facilmente notado na BR-101, onde em regiões serranas, a média de acidente pode ser até quarenta vezes a média em regiões não serranas \cite{acidentes}, fazendo desta rodovia, a mais cara em custos de acidentes no ano de 2017 \cite{anuario_rodoviario}.


\section{Definição do Problema}

<<<<<<< HEAD
Em trechos de curvas fechadas não há visibilidade do outro lado da estrada, podendo um motorista menos atento se envolver num acidente. Segundo os dados do (ref) no Brasil o segundo maior índice de acidentes ocorrem em trechos de curvas, além de ser o segundo tipo de trecho onde ocorre acidentes com vítimas. Ainda segundo (ref) a BR 101 é uma das rodovias mais perigosas do Brasil, no ano de 2017 foram registrados 4.704 acidentes com 788 vítimas fatais. 

\section{Proposta}

Este trabalho propõe-se a desenvolver um sistema de radares, nos quais deverão informar ao condutor, por meio de sinal luminoso, que um outro condutor também fará a curva, desta maneira alertando os motoristas para a redução de velocidade. Caso o condutor ultrapasse o limite da rodovia, será identificado a placa do motorista, por meio processamento em um servidor, para aplicação da multa correspondente a esta infração de trânsito.


=======
Este trabalho se propõe a desenvolver um sistema de radares, nos quais deverão informar ao condutor, por meio de sinal luminoso, que um outro condutor também fará a curva, desta maneira alertando os motoristas para a redução de velocidade. Caso o condutor ultrapasse o limite da rodovia, será identificado a placa do motorista, por meio processamento em uma central, para aplicação da multa correspondente a esta infração de trânsito.

\section{Justificativa}
>>>>>>> 2d3811d61b78f6dcc1fbfa71a935a083f702805d
\section{Objetivo Geral}

Desenvolvimento de um sistema de radares de efeito Doppler para detecção de carros em curva.


\subsection{Objetivos específico}