\chapter[Metodologia]{Metodologia}

Foi decidido que para o gerenciamento da equipe a metodologia ágil, \emph{Scrum}, seria utilizada. Foi feito um levantamento dos rituais agéis do \emph{Scrum}
e por ser um \emph{framework}, a equipe decidiu em aplicar apenas o que iria agregar valor ao time e ao projeto.

\section{Ritos do \emph{Scrum}}

\subsection{\emph{Sprint}}

No \emph{Scrum}, os projetos são dividos em ciclos chamados de {\emph{Sprints}. O {\emph{Sprint} representa um {\emph{time box} dentro do qual um conjunto 
de atividades deve ser executado. Foi definido que as {\emph{Sprints} do nosso projeto teriam a duração de uma semana. 

\subsection{Planejamento da \emph{Sprint}}

O planejamento da {\emph{Sprint} é uma reunião na qual estão presentes o {\emph{Product Owner}, o {\emph{Scrum Master} e todo o time, 
bem como qualquer pessoa interessada que esteja representando a gerência ou o cliente.

O objetivo desse planjemanento é determinar o que poderá ser entregue na {\emph{Sprint} que se inicia (criação do {\emph{Backlog da Sprint});

Esse planjemento deve ser negociado entre o {\emph{Product Owner} e restante da equipe, respeitando a capacidade projetada e a performance passada deste
estimar o esforço necessário para entregar as histórias do \emph{backlog}, através dificuldade dessas tarefas. 

Time Box: 30 minutos (20 minutos de planejamento nos subgrupos(por engenharia) e 10 minutos geral).

\subsection{\emph{Daily Meeting}}

Por conta da quantidade de membros e a contexto da matéria e da faculdade, foi decidido que a daily não irá ocorrer todos os dias com todos os membros presentes. 
Ao invés disso, cada sub-time irá realizar a sua \emph{daily} normalmente e cada líder dessa equipe irá atuar como o {\emph{Scrum Master}
nesse momento para identificar os impedimentos e ajudar na priorização do trabalho.

Os líderes de cada subgrupo deverão fazer uma daily entre eles e o coordenador geral para que o alinhamento geral funcione. 

Cada membro deverá responder as trés perguntas abaixo, sobre sua participação no andamento da \emph{Sprint}:

\begin{itemize}
    \item O que foi feito pelo membro no dia anterior para ajudar a desenvolver na Sprint?
    \item O que será feito pelo membro no dia atual para ajudar no desenvolvimento na Sprint?
    \item Houve algum impedimento para o membro que impossibilitou ele ajudar o restante da equipe na Sprint? (gerenciamento de riscos)
\end{itemize}

Time Box: 15 minutos

\subsection{Retrospectiva da\emph{Sprint}}

Ao final de cada \emph{Sprint} é feito uma retrospectiva. Durante esta reunião, o time mostra o que foi alcançado durante a \emph{Sprint} passada. 
Tipicamente, isso tem o formato de um \emph{demo} das novas funcionalidades. Os participantes da retrospectiva tipicamente incluem o \emph{Product Owner}, 
o Time, o \emph{Scrum Master}, gerência, clientes e demais integrantes do projetos.

O projeto é avaliado em relação aos objetivos do Sprint, determinados durante o Planejamento Sprint. Idealmente, a equipe completou cada um dos itens do 
\emph{Product Backlog} trazidos para fazer parte da Sprint, mas o importante mesmo é que a equipe atinja o objetivo geral da Sprint.
Dentro da \emph{Sprint Review}será feito um levantamento e servirá para identificar o que funcionou bem, o que pode ser melhorado e que ações serão tomadas para melhorar.

Time Box: 20 minutos    

\subsection{\emph{Product Backlog}}

O {\emph{Product Backlog} é uma lista contendo todas as funcionalidades desejadas para um produto. O conteúdo desta lista é definido pelo {\emph{Product Owner}. 
O {\emph{Product Backlog} não precisa estar completo no início de um projeto. Pode-se começar com tudo aquilo que é mais óbvio em um primeiro momento. 
Com o tempo, o {\emph{Product Backlog} cresce e muda à medida que se aprende mais sobre o produto e seus usuários.

Durante o planejamento da {\emph{Sprint}, o {\emph{Product Onwer} prioriza os itens do {\emph{Product Backlog} e os descreve para a equipe. A equipe então determina que itens será 
capaz de completar durante a {\emph{Sprint} que está por começar. Tais itens são, então, transferidos do {\emph{Product Backlog} para o {\emph{Sprint Backlog}. 
Ao fazer isso, a equipe quebra cada item do {\emph{Product Backlog} em uma ou mais tarefas. Isso ajuda a dividir o trabalho entre os membros 
da equipe. Podem fazer parte do {\emph{Product Backlog} tarefas técnicas ou atividades diretamente relacionadas às funcionalidades solicitadas.


\section{Papéis na Equipe}

\subsection{\emph{Product Owner}}

O \emph{Product Owner} tem a responsabilidade de gerir o \emph{backlog} do produto e por garantir o valor do trabalho realizado pelo time. Adaptamos o papel 
do \emph{Product Owner} e para alcançarmos maior qualidade naquilo que é entregado optamos por ter um \emph{Producto Onwer} por subgupo, ou seja, por engenharia.

Atribuições:
\begin{itemize}
    \item Manter o \emph{Backlog} do Produto e garante que ele está visível para todos;
    \item Informar a todos quais itens têm a maior prioridade, de forma que todos sabem em que se irá trabalhar;
    \item Definir e priorizar os itens do \emph{Backlog} do Produto;
    \item Valor de negócio;
    \item Visão de negócio;
    \item Canvas.
\end{itemize}

\subsection{\emph{Scrum Master}}

Atribuições:
\begin{itemize}
    \item Ajudar todos do Time \emph{Scrum} a entenderem a teoria, prática, regras e valores do \emph{Scrum};
    \item Servir ao \emph{Product Owner}, auxiliando de diversas formas, tais como: Gerir de maneira eficiente o \emph{Backlog} do produto;
    \item Fazer com que todos do Time entendam ao máximo os itens do Backlog do Produto.
    \item Remover impedimentos ao progresso de todos da equipe;
    \item Instrui-lo em auto-organização e a serem multifuncionais;
    \item Documentar cada Sprint;
    \item Determinar e analisar as métricas e indicadores utilizados para acompanhar o progresso de toda a equipe.
\end{itemize}
